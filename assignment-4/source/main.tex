\documentclass{article}
\usepackage{listings}
\usepackage{xcolor}

\title{C code}
\author{maheshvasimalla }
\date{December 2020}

\definecolor{mGreen}{rgb}{0,0.6,0}
\definecolor{mGray}{rgb}{0.5,0.5,0.5}
\definecolor{mPurple}{rgb}{0.58,0,0.82}
\definecolor{backgroundColour}{rgb}{0.95,0.95,0.92}

\lstdefinestyle{CStyle}{
    backgroundcolor=\color{backgroundColour},   
    commentstyle=\color{mGreen},
    keywordstyle=\color{magenta},
    numberstyle=\tiny\color{mGray},
    stringstyle=\color{mPurple},
    basicstyle=\footnotesize,
    breakatwhitespace=false,         
    breaklines=true,                 
    captionpos=b,                    
    keepspaces=true,                 
    numbers=left,                    
    numbersep=5pt,                  
    showspaces=false,                
    showstringspaces=false,
    showtabs=false,                  
    tabsize=2,
    language=C
}

\begin{document}

\maketitle

\begin{lstlisting}[style=CStyle]
//Code written on December 30, 2020

//Revised  December 30, 2020

// by maheshvasimalla

//This program implements the incremental decoder using boolean logic in C

#include <stdio.h>

//The  main function
int main(void)
{

//2 bits = 1 baud
//4 bits = 1 nibble
//8 bits = 1 byte

//unsigned char takes input as 1 byte

unsigned char  Z=0x01,Y=0x00,X=0x00,W=0x01;//inputs in hex	
unsigned char one = 0x01;//used for displaying the output in bit
unsigned char A,B,C,D;//outputs

D = (W&X&Y&(~Z))|((~W)&(~X)&(~Y)&Z);//Boolean function for D
B=((~Z)&(~Y)&(~X)&W)|((~Z)&(~Y)&X&(~W))|((~Z)&Y&(~X)&W)|((~Z)&Y&X&(~W));
C=((~Z)&(~Y)&X&W)|((~Z)&Y&(~X)&(~W))|((~Z)&Y&(~Z)&W)|((~Z)&Y&X&(~W));
A = ((~W)&(~X)&(~Y)&(~Z))|((~W)&(X)&(~Y)&(~Z))|((~W)&(~X)&Y&(~Z))|((~W)&X&Y&(~Z))|((~W)&(~X)&(~Y)&(Z));
//Boolean function for A

printf("%x%x%x%x",one&Z,one&Y,one&X,one&W);//Iutput ZYXW
printf(" ");
printf("%x%x%x%x\n" ,one&D,one&C,one&B,one&A);//Output DCBA
return 0;
}

\end{lstlisting}
\end{document}
